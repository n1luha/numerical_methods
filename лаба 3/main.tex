\documentclass[12pt]{article}
\usepackage{graphicx}
\usepackage{float}
\usepackage{wrapfig}
\usepackage{cmap}
\usepackage{hyperref} 
\usepackage[russian]{babel}
\usepackage[a4paper, total={7in, 10in}]{geometry}
\usepackage{verbatim}
\usepackage{booktabs}
\usepackage{listings}
\usepackage[russian]{babel}
\usepackage[utf8]{inputenc}
\usepackage{amssymb, amsmath}
\usepackage[warn]{mathtext}
\usepackage[perpage]{footmisc}
\makeatletter
\renewenvironment{titlepage} {
\thispagestyle{empty}
}
\makeatother
\begin{document}
\thispagestyle{empty}
\begin{titlepage}
    \begin{center}
            \large Санкт-Петербургский политехнический университет Петра Великого\\
        \large Физико-механический институт \\
        \large Направление подготовки\\
        \large "01.03.03. Механика и математическое моделирование"\\[3cm]
        \huge Дисциплина "Численные методы"\\[0.5cm] 
        \large Отчет по лабораторной работе №3\\[0.1cm]
        \large {Решение интегралов с помощью квадратурных формул Ньютона-Котеса.}\\[5cm]
    \end{center}
        \begin{flushright} 
            \begin{minipage}{0.25\textwidth} 
                \begin{flushleft} 
                    \large\textbf{Работу выполнил:}\\
                    \large Нилов И.Р.\\
                    \large {Группа:} 5030103/30001\\		
                    \large \textbf{Преподаватель:}\\
                    \large Козлов К.Н.
                \end{flushleft}
            \end{minipage}
        \end{flushright}
    \vfill 
    \begin{center}
    \large Санкт-Петербург\\
    \large \the\year 
    \end{center} 
\end{titlepage} %
\newpage
\tableofcontents
\newpage

\begin{center}
    \section{Формулировка задачи и ее формализация}
\end{center}

\subsection{Формализация задачи}
Вычислить интеграл с помощью квадратурной формулы Ньютона-Котеса, построенной при помощи метода средних прямоугольников.
\subsection{Поставленные задачи}
\begin{itemize}
    \item Ручной расчет провести для 1, 2 и 4х разбиений отрезка,      оценить точность вычисления интеграла.
    
    \item Написать программу для вычисления приближенного значения     интеграла с заданной точностью с помощью обобщенной формулы, для достижения заданной точности использовать правило Рунге.             
    \item График зависимости фактической ошибки от заданной точности, отметить линию биссектрисы. Проверить заданную точность: $10^{-1}$, $10^{-2}$, ..., $10^{-16}$.

    \item График зависимости числа итераций от заданной точности.
    
    \item График фактической ошибки от длины отрезка разбиения, использовать логарифмический масштаб по основанию 2. По графику определить порядок точности применяемой формулы и вычислить константу.
\end{itemize}

\begin{center}
    \section{Алгоритм и условия применимости}

\end{center}

\subsection{Условия применимости}
\begin{enumerate}
    \item $p(x) \equiv1$, где p - весовая функция
    \item $x_k=a+(i+0.5)h, h=\frac{b-a}{n}$
\end{enumerate}

\subsection{Алгоритм}
Строим квадратурную формулу для функции f на отрезке [a,b] по данной формуле

\begin{equation*}
    \int_a^bf(x)dx\approx h \sum\limits_{i=0}^{n-1}f(x_i)
\end{equation*}
где $h=\frac{b-a}{n}$ - шаг разбиения, n - количество разбиений, $x_i=a+(i+0.5)h$ - точки разбиения.
\begin{center}
    \section{Предварительный анализ задачи}
\end{center}
25 вариант включает себя интегрирование следующей функции:
\begin{equation*}
    x^5 - 9.2x^3 + 5.5x^2 - 7x 
\end{equation*}
Для интегрирования был выбран отрезок [0, 2]. Было вычислено 'точное' значение интеграла, равное -25.4666666667.

\vspace{1cm}

\begin{center}
    \section{Тестовый пример}
\end{center}

\begin{figure}[H]
    \centering
    \includegraphics[width=0.6\linewidth]{photo_2025-03-21_11-12-24.jpg}
    \caption{Точное значение интеграла}
    \label{fig:enter-label}
\end{figure}

\begin{figure}[H]
    \centering
    \includegraphics[width=0.8\linewidth]{photo_2025-03-18_18-22-39.jpg}
    \caption{Тестовый пример}
    \label{fig:enter-label}
\end{figure}

\begin{center}
    \section{Модульная структура программы}
\end{center}
\begin{lstlisting}
    def midpoint_rule(a, b, n):
\end{lstlisting} - функция принимает a и b - границы отрезка, n - количество разбиений, возвращает integral - приближенное значение интеграла.

\begin{lstlisting}
    def midpoint_rule_with_rung(a, b, eps, max_iter):
\end{lstlisting} - функция принимает a и b - границы отрезка, eps - заданную точность, maxiter - максимальное количество итераций, возвращает intnew - приближенное значение интеграла, вычисленное с заданной точностью.

\begin{lstlisting}
    def add_perturbation(k, perturbation_percent):
\end{lstlisting} - функция принимает k - значение константы, perturbationpercent - процент возмущения, который нужно внести в константу, возвращает новое значение константы, полученное после добавления случайного возмущения.

\begin{center}
    \section{Численный анализ}
\end{center}

В ходе выполенения лабораторной работы было проведено исследование зависимости фактической ошибки от заданной точности. Набор точностей: $10^{-1}, 10^{-2}, ..., 10^{-10}.$ Ниже приведён график зависимости фактической ошибки от заданной точности.

\begin{figure}[H]
    \centering
    \includegraphics[width=0.7\linewidth]{2025-03-16_18-08-29.png}
    \caption{Зависимость фактической ошибки от заданной точности}
    \label{fig:enter-label}
\end{figure}

График, построенный по результатам эксперимента находится ниже линии биссектрисы, что говорит о том, что метод достигает заданной точности.

Также было проведено исследование зависимости числа итераций от заданной точности.
Ниже приведён график, построенный по результатам эксперимента.

\begin{figure}[H]
    \centering
    \includegraphics[width=0.6\linewidth]{2025-03-16_18-09-31.png}
    \caption{Зависимость числа итераций от заданной точности}
    \label{fig:enter-label}
\end{figure}

По графику видно, что для достижения большей точности требуется большее число итераций.

Было проведено исследование зависимости фактической ошибки от длины отрезка разбиения. Набор количеств разбиений: $5 * 2, 5 * 2^2, 5 * 2^3, 5 * 2^4, ... 5 * 2^{10}$.


\begin{figure}[H]
    \centering
    \includegraphics[width=0.6\linewidth]{2025-03-18_18-36-31.png}
    \caption{Зависимость фактической ошибки от длины отрезка разбиения}
    \label{fig:enter-label}
\end{figure}

По графику видно, что с уменьшением длины отрезка разбиения уменьшаяется фактическая ошибка. По данному графику можно определить порядок точности применяемого метода и вычислить константу.

Был вычислен порядок точности $\approx2.01226$, который соответствует порядку метода. Константу определим, как $\approx0.37403$.

В качестве дополнительного исследования были внесены погрешности 1, 2 и 3 процента в вычисление второго коэффициента полинома. Была вычислена относительная ошибка для каждого значения максимального возмущения и построен график.

\begin{figure}[H]
    \centering
    \includegraphics[width=0.6\linewidth]{2025-03-16_18-21-58.png}
    \caption{Зависимость относительной ошибки от уровня возмущения}
    \label{fig:enter-label}
\end{figure}

\begin{center}
    \section{Выводы}
\end{center}

В ходе лабораторной работы были проведены исследования метода средних прямоугольников вычисления определённых интегралов. В результате исследований был построен график зависимости фактической ошибки от заданной точности, график зависимости числа итераций от заданной точности и график зависимости фактической ошибки от длины отрезка разбиения. В ходе исследований было показано, что порядок метода равен 2.

Также была исследована зависимость относительной ошибки от уровня возмущения. Построенный график показывает, что относительная ошибка не превышает процент возмущения входных данных, что свидетельствует о хорошей устойчивости.


\end{document}